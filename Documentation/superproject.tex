\documentclass[a4paper]{scrreprt}

%% Language and font encodings
\usepackage[utf8]{inputenc}
\usepackage[T1]{fontenc}

%% Sets page size and margins
\usepackage[a4paper,top=3cm,bottom=2cm,left=3cm,right=3cm,marginparwidth=1.75cm]{geometry}

%% Useful packages
\usepackage{amsmath}
\usepackage{graphicx}
\usepackage[colorinlistoftodos]{todonotes}
\usepackage[colorlinks=true, allcolors=blue]{hyperref}

\title{The Blacksmith Journey (TEMPORARY)}
\subtitle{"WHAT IF YOU PLAY A FANTASY ADVENTURE FROM THE BLACKSMITH SIDE?"}
\author{Marco "Markof" Fabiani}


\begin{document}
\maketitle

\null\vfill
\noindent
Game Design Document\\ 
Version v0.1.1, Oct 2019\\
\newpage


\tableofcontents

% ______________________
% chapter Overview
% ______________________
\chapter{Overview}

\section{Project Description}
The games market is so plenty of fantasy titles with various genres: RPG, Action RPG, Hack 'n Slash, RTS etc...\\
This is a very exciting setting, with several possibilities of game mechanics, enchanted items, cool texture weapons, magnificents spells and many more, but we also have to admit that is a kind of scenario 
poor of recurring ideas nowadays.\\
Since we always control an hero character, or a group of them, joining various adventures with many legs on armories to improve our equipments, why don't we manage an armory this time? Why can't we see a fantasy plot from the blacksmith point of view?\\
Today, thanks to this title, we can really do that! 

\section{Theme/Setting/Genre}
\textbf{The Blacksmith Journey} will be a pure \textbf{management game} on a fantasy world. Despite the unrealistic setting, the purpose of the game is to realise many structures and tools close to the real ones used during medieval age.\\
Of course you'll have spells, supernatural abilities and other magical stuff all around your armory.

\section{Targeted Platforms}
\textbf{PC} of course, maybe \textbf{Android}.

\section{Core Gameplay}
For now I think it's good to develop a \textbf{story mode}, with a beginning and an end. In the near future I'd like to enhance this story with a \textbf{non-linear} structure, based on the populations choices.\\
Multiplayer mode? Maybe... if it's a great success!

\section{Influence}
On the graphic point of view, character models and colours like \textbf{Clash Royale}, maybe even smoother. UI and text too.\\
I'd like to keep only the usability realism of the various structures, from the gameplay point of view. But graphically I want to give them a bit of cartoon mood.\\
An easygoing mood like a very old genre masterpiece like \textbf{Theme Hospital}.\\
Music and sounds few original, maybe inspired by parody movies like \textbf{Robin Hood: Men in Tights}.

% ______________________
% chapter Game Design
% ______________________

\chapter{Game Design} 

\section{Referral Guidelines}
This is a little recap of the detailed Game Design. Here it'll be various references to other documents that illustrates more in depth some of this following points. 

\subsection{Mood And Emotions}
The game mood is a bit epic with a lot of weapons, NPCs etc... but not so serious. It's important to keep this game like easy and nice mood, even in the most cruel moments.

\subsection{Story}
(PER ORA LO SCRIVO IN ITALIANO) La storia, per quanto sia secondaria ai fini del gameplay, è incisiva nel momento in cui si devono effettuare delle scelte in termini di obiettivi e missioni da completare. Le tre grandi fazioni coinvolte dovranno realizzare i loro obiettivi nel mondo di gioco, attraverso accordi, battaglie, complotti e diplomazia con la partecipazione di altre sei fazioni minori, ognuna con la sua peculiarità.\\
Sullo sfondo

\subsection{World/Environment}
qual è l'ambientazione del gioco. Inoltre, se utile, inserire la mappa dell'ambiente o del mondo di gioco

\subsection{Main Objectives}
quali sono gli obiettivi principali del gioco

\subsection{Character in Game}
chi sono i personaggi del gioco

\subsection{Main Objects}
gli oggetti peculiari del gioco

\subsection{Core Mechanics}
descrizione delle meccaniche centrali

% ______________________
% chapter Technical Aspects
% ______________________

\chapter{Technical Aspects}
sezione dedicata agli elementi più tecnici sulle meccaniche ed i controlli di gioco

\section{Front End/Screens}

\begin{itemize}
\item Logos/Fonts/Images
\item Splash Screen
\item Title Screen
\item Main Menu
\item Options
\item Credits
\end{itemize}

\section{Controls}
descrizione approfondita dei comandi di gioco

\section{Mechanics}
descrivere eventuali meccaniche di gioco originali o peculiari del progetto

% ______________________
% chapter Level Design
% ______________________

\chapter{Level Design}
be specific about the core game features 

\section{Themes}
descrivere stato d'animo, environment ed oggetti peculiari all'interno di ciascun livello previsto

\section{Game Flow}
flusso del gioco attraverso i livelli e le mappe

% ______________________
% chapter Development Systems/Tools
% ______________________

\chapter{Development Systems/Tools}
quali applicazioni e strumenti, nel dettaglio, vengono adottati nella creazione degli asset grafici, sonori e della produzione dei codici sorgenti

% ______________________
% chapter Graphics
% ______________________

\chapter{Graphics}

\section{Style Attributes}
tipo di colori, stili grafici, camera, post-production ecc...

\section{Graphics Needed}

\begin{itemize}
\item 2D (texture, immagini ecc...)
\item 3D
\item Animations
\item Lights
\end{itemize}

% ______________________
% chapter Sounds/Musics
% ______________________

\chapter{Sounds/Musics}

\section{Style Attributes}
come per la grafica

\section{Sounds Needed}
effetti sonori, feedback su interazioni

\section{Musics Needed}

% ______________________
% chapter Development
% ______________________

\chapter{Development}

\section{Abstract Classes}

\section{Derived Classes}

% ______________________
% chapter Monetization Model
% ______________________

\chapter{Monetization Model}
tipo di monetizzazione (premium, paid alpha/beta/final, early access, micro-transazioni, subscriptions ecc...). Eventuale link al documento di modello.

% ______________________
% chapter Schedule Timeline
% ______________________

\chapter{Schedule Timeline}
Suddivisione milestone e tempi. Eventuali link/screenshots a diagrammi di Gantt

\begin{table}[h]
\centering
\begin{tabular}{|l|l|l|}
\hline
Milestone & Description & Date \\\hline
& Official Start Date & 01.12.... \\
1 & Milestone Description ..  & 01.12.... \\
2 & Milestone Description ..  & 01.01.... \\
3 & Milestone Description ..  & 01.03.... \\
& End of Project & 01.04.... \\
\hline
\end{tabular}
\caption{\label{tab:schedule}Example Schedule.}
\end{table}

\chapter{Appendix A - THE LEGENDARY DIVSIONS BOOK}
Nel vasto regno di \textbf{Sutrem}, dopo un sanguinoso susseguirsi di battaglie, la situazione geopolitica del territorio vede l'equilibrio di tre principali forze in competizione per prevalere come grande potenza dominante: la \textbf{Repubblica Democratica di Jerelath}, il \textbf{regno di Filiri} (plutocrazia) ed il \textbf{Sacro Impero di Zaorwen}.\\
In tutto il territorio, oltre a queste tre entità, sono presenti altre cinque fazioni più o meno indipendenti dalla loro influenza. Periodicamente, dalla misteriosa \textbf{Torre di Puro Slaemt} (da "male" in islandese).



\end{document}