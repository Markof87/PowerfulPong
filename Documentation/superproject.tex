\documentclass[a4paper]{scrreprt}

%% Language and font encodings
\usepackage[utf8]{inputenc}
\usepackage[T1]{fontenc}

%% Sets page size and margins
\usepackage[a4paper,top=3cm,bottom=2cm,left=3cm,right=3cm,marginparwidth=1.75cm]{geometry}

%% Useful packages
\usepackage{amsmath}
\usepackage{graphicx}
\usepackage[colorinlistoftodos]{todonotes}
\usepackage[colorlinks=true, allcolors=blue]{hyperref}

\title{The Blacksmith Journey (TEMPORARY)}
\subtitle{"WHAT IF YOU PLAY A FANTASY ADVENTURE FROM THE BLACKSMITH SIDE?"}
\author{Marco "Markof" Fabiani}


\begin{document}
\maketitle

\null\vfill
\noindent
Game Design Document\\ 
Version v0.1.1, Oct 2019\\
\newpage


\tableofcontents

% ______________________
% chapter Overview
% ______________________
\chapter{Overview}

\section{Project Description}
The games market is so plenty of fantasy titles with various genres: RPG, Action RPG, Hack 'n Slash, RTS etc...\\
This is a very exciting setting, with several possibilities of game mechanics, enchanted items, cool texture weapons, magnificents spells and many more, but we also have to admit that is a kind of scenario 
poor of recurring ideas nowadays.\\
Since we always control an hero character, or a group of them, joining various adventures with many legs on armories to improve our equipments, why don't we manage an armory this time? Why can't we see a fantasy plot from the blacksmith point of view?\\
Today, thanks to this title, we can really do that! 

\section{Theme/Setting/Genre}
\textbf{The Blacksmith Journey} will be a pure \textbf{management game} on a fantasy world. Despite the unrealistic setting, the purpose of the game is to realise many structures and tools close to the real ones used during medieval age.\\
Of course you'll have spells, supernatural abilities and other magical stuff all around your armory.

\section{Targeted Platforms}
\textbf{PC} of course, maybe \textbf{Android}.

\section{Core Gameplay}
For now I think it's good to develop a \textbf{story mode}, with a beginning and an end. In the near future I'd like to enhance this story with a \textbf{non-linear} structure, based on the populations choices.\\
Multiplayer mode? Maybe... if it's a great success!

\section{Influence}
On the graphic point of view, character models and colours like \textbf{Clash Royale}, maybe even smoother. UI and text too.\\
I'd like to keep only the usability realism of the various structures, from the gameplay point of view. But graphically I want to give them a bit of cartoon mood.\\
An easygoing mood like a very old genre masterpiece like \textbf{Theme Hospital}.\\
Music and sounds few original, maybe inspired by parody movies like \textbf{Robin Hood: Men in Tights}.

% ______________________
% chapter Game Design
% ______________________

\chapter{Game Design} 

\section{Referral Guidelines}
This is a little recap of the detailed Game Design. Here it'll be various references to other documents that illustrates more in depth some of this following points. 

\subsection{Mood And Emotions}
The game mood is a bit epic with a lot of weapons, NPCs etc... but not so serious. It's important to keep this game like easy and nice mood, even in the most cruel moments.

\subsection{Story}
(PER ORA LO SCRIVO IN ITALIANO) La storia, per quanto sia secondaria ai fini del gameplay, è incisiva nel momento in cui si devono effettuare delle scelte in termini di obiettivi e missioni da completare. Le tre grandi fazioni coinvolte dovranno realizzare i loro obiettivi nel mondo di gioco, attraverso accordi, battaglie, complotti e diplomazia con la partecipazione di altre cinque fazioni minori, ognuna con la sua peculiarità.\\
Sullo sfondo la minaccia dei demoni al servizio di Karathal l'Immortale può influire sulle situazioni di gioco, spingendo il giocatore a dover individuare soluzioni alternative agli eventi in corso.\\
Nell'Appendice A è riportata l'intera lore delle fazioni coinvolte nel mondo di gioco.

\subsection{World/Environment}
qual è l'ambientazione del gioco. Inoltre, se utile, inserire la mappa dell'ambiente o del mondo di gioco

\subsection{Main Objectives}
quali sono gli obiettivi principali del gioco

\subsection{Character in Game}
chi sono i personaggi del gioco

\subsection{Main Objects}
CIBO, MATERIALI, SOSTANZE.

Modalità razioni di cibo (N.B: proteine, carboidrati, zuccheri, grassi ecc...)

\subsection{Core Mechanics}
descrizione delle meccaniche centrali

% ______________________
% chapter Technical Aspects
% ______________________

\chapter{Technical Aspects}
sezione dedicata agli elementi più tecnici sulle meccaniche ed i controlli di gioco

\section{Front End/Screens}

\begin{itemize}
\item Logos/Fonts/Images
\item Splash Screen
\item Title Screen
\item Main Menu
\item Options
\item Credits
\end{itemize}

\section{Controls}
descrizione approfondita dei comandi di gioco

\section{Mechanics}
descrivere eventuali meccaniche di gioco originali o peculiari del progetto

% ______________________
% chapter Level Design
% ______________________

\chapter{Level Design}
be specific about the core game features 

\section{Themes}
descrivere stato d'animo, environment ed oggetti peculiari all'interno di ciascun livello previsto

\section{Game Flow}
flusso del gioco attraverso i livelli e le mappe

% ______________________
% chapter Development Systems/Tools
% ______________________

\chapter{Development Systems/Tools}
quali applicazioni e strumenti, nel dettaglio, vengono adottati nella creazione degli asset grafici, sonori e della produzione dei codici sorgenti

% ______________________
% chapter Graphics
% ______________________

\chapter{Graphics}

\section{Style Attributes}
tipo di colori, stili grafici, camera, post-production ecc...

\section{Graphics Needed}

\begin{itemize}
\item 2D (texture, immagini ecc...)
\item 3D
\item Animations
\item Lights
\end{itemize}

% ______________________
% chapter Sounds/Musics
% ______________________

\chapter{Sounds/Musics}

\section{Style Attributes}
come per la grafica

\section{Sounds Needed}
effetti sonori, feedback su interazioni

\section{Musics Needed}

% ______________________
% chapter Development
% ______________________

\chapter{Development}

\section{Abstract Classes}

\section{Derived Classes}

% ______________________
% chapter Monetization Model
% ______________________

\chapter{Monetization Model}
tipo di monetizzazione (premium, paid alpha/beta/final, early access, micro-transazioni, subscriptions ecc...). Eventuale link al documento di modello.

% ______________________
% chapter Schedule Timeline
% ______________________

\chapter{Schedule Timeline}
Suddivisione milestone e tempi. Eventuali link/screenshots a diagrammi di Gantt

\begin{table}[h]
\centering
\begin{tabular}{|l|l|l|}
\hline
Milestone & Description & Date \\\hline
& Official Start Date & 01.12.... \\
1 & Milestone Description ..  & 01.12.... \\
2 & Milestone Description ..  & 01.01.... \\
3 & Milestone Description ..  & 01.03.... \\
& End of Project & 01.04.... \\
\hline
\end{tabular}
\caption{\label{tab:schedule}Example Schedule.}
\end{table}

\chapter{Appendix A - THE LEGENDARY DIVSIONS BOOK}
Nel vasto regno di \textbf{Sutrem}, dopo un sanguinoso susseguirsi di battaglie, la situazione geopolitica del territorio vede l'equilibrio di tre principali forze in competizione per prevalere come grande potenza dominante: la \textbf{Repubblica Democratica di Jerelath}, il \textbf{Ducato di Filiri} (plutocrazia) ed il \textbf{Sacro Impero di Zaorwen}.\\
In tutto il territorio, oltre a queste tre entità, sono presenti altre cinque fazioni più o meno indipendenti dalla loro influenza. Dalla misteriosa \textbf{Torre di Puro Slaemt} (da "male" in islandese), orde di demoni al servizio di \textbf{Karathal l'Immortale} occupano o assediano periodicamente porzioni di territorio con l'intento di strapparle agli esseri umani.\\

\section{I tre grandi elementi}
Durante i sette cicli delle \textbf{Campagne di Dominio} che hanno attraversato tutto il periodo antico, le fazioni accennate in precedenza hanno raggiunto un equilibrio ed un lungo periodo di rapporti diplomatici grazie alla spartizione dei tre grandi elementi del regno: il \textbf{Mana} per controllare flussi di magia e realizzare incantesimi, il \textbf{Lami} per alimentare le batterie degli apparati tecnologici più avanzati, e la \textbf{Fairiu}, prezioso liquido per coniare monete.\\
Ognuna delle fazioni è specializzata nella raccolta e sfruttamento di questi importanti materiali. Il controllo su ciascuno di questi elementi consente di ottenere e spartire con facilità il resto delle risorse più comuni nel regno.\\
Vediamo insieme nel dettaglio i tre grandi territori. 

\subsection{Repubblica Democratica di Jerelath}
Il primo esempio di repubblica mai apparsa al mondo. Al termine delle Campagne di Dominio, grazie anche al supporto del Concilio di Zaorwen, il territorio di Jerelath fu liberato dalla opprimente presa del mostruoso dittatore \textbf{Druktian il Senzasangue (NOME PROVVISORIO)}, uno dei tre grandi \textbf{Assistenti di Karathal}.\\ 
Scoperta l'origine del Lami e le sue applicazioni per generare energia cinetica pulita ed efficiente, gli scienziati ribelli realizzarono una vera e propria bomba a frammentazione con la quale abbatterono il dittatore. Fondamentale l'utilizzo di metallo incantato per l'involucro della bomba, messo a disposizione dai membri del Concilio di Zaorwen, finanziati a loro volta dai ricchi nobili del Ducato di Filiri.\\
Negli anni a seguirsi, dopo aver appena trascorso diversi lustri di sofferenza, gli abitanti di Jerelath sentirono come naturale reazione l'esigenza di una reale forma di partecipazione collettiva nel paese, includendo anche i più deboli e le minoranze più oppresse. Istituirono così una forma di governo a rappresentanza, con il \textbf{Senato dei Dieci Anziani} a scrivere e promulgare proposte di legge, ed una \textbf{Seduta} di 400 cittadini eletti dal popolo, ripartiti equamente per fasce di età a cui corrisponde il loro stesso elettorato: per esempio, nella fascia tra i 35 ed i 50 anni esistono solo rappresentanti che appartengono anagraficamente ad essa, e soltanto i cittadini di quelle età possono esprimere una preferenza verso di loro. 
Il loro ruolo è quello di decidere se acconsentire a portare avanti una proposta di legge, modificarla o rigettarla completamente.\\
Nella vita politica non sono esclusi nemmeno i bambini, che hanno una propria fascia fino ai 18 anni equamente rappresentata nella Seduta.\\
\\
A livello economico i jerelathiani possono contare su una tecnologia decisamente più avanzata rispetto a qualsiasi altra popolazione, grazie ai loro prodigi della meccanica alimentati con il Lami. Quest'ultimo, una volta estratto e raffinato, viene confezionato in enormi contenitori stipati presso la \textbf{Roccaforte di Krapfut}, da cui un vero e proprio sistema di tubature parallelo a quello della distribuzione dell'acqua si dirama servendo tutte le fabbriche più importanti del territorio. Una parte del liquido viene messo da parte per l'esportazione al Ducato di Filiri, pronto a pagare profumatamente svariate centinaia di litri.\\
La particolarità dei materiali derivati e la complessità dei componenti utilizzati nei processi produttivi rendono Jerelath un partner commerciale molto difficile: è possibile farsi strada nel loro mercato soltanto con supporti per macchine utensili, pezzi unici di artigianato ed armi incantate, queste ultime sempre più rare con l'inasprirsi dei rapporti con Zaorwen. Un buon fabbro che abbia intenzioni di condurre affari con loro deve essere un maestro di efficienza e puntualità nelle consegne, preciso nei dettagli dei componenti e padroneggiare eccellenti abilità di produzione di metalli derivati.\\
La caduta del regime ha consentito una incredibile crescita del ceto medio, che ha iniziato a specializzarsi prevalentemente in ambiti scientifici e tecnologici. Alcuni hanno potuto accumulare grandi fortune diventando fondatori di diverse attività importanti nel territorio, altri hanno raggiunto un livello di conoscenza quasi accademica, fondando la prima università del regno di Sutrem a \textbf{Thearin}, la capitale di Jerelath. L'iniziativa della Seduta, incalzata dal Senato, ha portato una considerevole tutela nei corsi di laurea ingegneristici (in particolare riguardo il trattamento dei materiali, facoltà che attira i figli dei nobili più importanti di Filiri) a scapito di altre materie molto meno in evidenza.\\
\\

ARCHITETTURA\\

\\
La regione è laica, ma la maggior parte dei cittadini tende a professarsi non credente. Le principali ricorrenze sono il \textbf{Mese della Liberazione} a giugno e la \textbf{Scoperta del Lami} il 14 dicembre.\\
Durante il Mese della Liberazione, la popolazione che esercita professioni tecnico-scientifiche è tenuta a lavorare solo per quattro ore, dedicando il resto della giornata ad attività più artistiche o manuali. Vengono inoltre esaltati i prodotti dell'agricoltura, solitamente considerata secondaria, con piccoli chioschi dedicati alle ricette più deliziose del posto.\\
La dieta tipica di Jerelath è piuttosto povera di carne. L'animale più gettonato per la cucina è il \textbf{Maiale Rosso}, più rapido e sfuggente del normale, ma con tagli molto più magri e digeribili.\\
\\
Jerelath è considerata all'avanguardia per l'attenzione alle minoranze ed i diritti civili estesi a tutti i cittadini. Tuttavia si è formato nel tempo un ampio solco economico e sociale tra i più vicini alle scienze, spesso eletti nella Seduta o addirittura nominati per il Senato, e tutti gli altri. Per quanto non ci siano oppressioni nei confronti di questi ultimi, le condizioni economiche di cui possono godere non consentono uno stile di vita privo di preoccupazioni.\\
Alcuni di essi, con posizioni più estreme, fanno parte degli \textbf{Amici di Druktian}, un movimento reazionario diventato quasi un vero e proprio culto religioso dove i membri auspicano un ritorno alla vecchia dittatura, -che a fronte di una maggiore povertà, garantiva merci basilari per tutti-, fino ad arrivare ad alcuni sporadici attentati terroristici.\\
Il Senato, contro lo stesso principio di libertà di parola assoluta, sta pensando di risolvere la questione bandendo il culto dal territorio.\\
Un altro problema più recente è la scarsa immigrazione che ha in parte bloccato la crescita economica: gli unici stranieri che giungono a Jerelath sono giovani che puntano a formarsi nelle università e successivamente, spesso e volentieri, rientrare nel loro luogo di provenienza. Scarseggiano gli arrivi di artigiani qualificati, agricoltori e vari intellettuali, che trovano posizioni decisamente più allettanti in altri territori.  

\subsection{Sacro Impero di Zaorwen}

\end{document}